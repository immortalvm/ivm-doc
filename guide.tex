\documentclass[a4paper,11pt]{article}
\usepackage[utf8]{inputenc}
\usepackage[T1]{fontenc}
\usepackage[includeheadfoot,margin=2cm]{geometry}
\usepackage{times}
\usepackage{booktabs}
\usepackage{xspace}
\frenchspacing
\renewcommand{\ttdefault}{cmtt}
\author{Thor Kristoffersen}
\date{\today}
\title{Guide to Building a Machine}
\newcommand{\PC}{\textbf{program counter}\xspace}
\newcommand{\SP}{\textbf{stack pointer}\xspace}
\newcommand{\TERM}{\textbf{terminated}\xspace}
\newcommand{\F}{\textbf{false}\xspace}
\newcommand{\T}{\textbf{true}\xspace}
\newcommand{\num}[1]{\texttt{#1}\xspace}
\newcommand{\hex}[1]{\num{#1}$_{\textup{\tiny\hskip-1ex 16}}$\xspace}
\newcommand{\bin}[1]{\num{#1}$_{\textup{\tiny\hskip-1ex 2}}$\xspace}
\newcommand{\comment}[1]{\begin{quote}[\textit{#1}]\end{quote}}
\newcommand{\op}[1]{#1}
\newcommand{\EXIT}    [1]{\op{\hex{00}}\xspace}
\newcommand{\NOP}     [1]{\op{\hex{01}}\xspace}
\newcommand{\JUMP}    [1]{\op{\hex{02}}\xspace}
\newcommand{\JUMPZERO}[1]{\op{\hex{03}}\xspace}
\newcommand{\SETSP}   [1]{\op{\hex{04}}\xspace}
\newcommand{\GETPC}   [1]{\op{\hex{05}}\xspace}
\newcommand{\GETSP}   [1]{\op{\hex{06}}\xspace}
\newcommand{\PUSHZ}   [1]{\op{\hex{07}}\xspace}
\newcommand{\PUSHB}   [1]{\op{\hex{08}}\xspace}
\newcommand{\PUSHS}   [1]{\op{\hex{09}}\xspace}
\newcommand{\PUSHI}   [1]{\op{\hex{0A}}\xspace}
\newcommand{\PUSHL}   [1]{\op{\hex{0B}}\xspace}
\begin{document}

\maketitle

\noindent
This film contains programs that can be executed on a machine.
The purpose of this part of the film is to guide you through the process of building this machine.
To understand this guide, you need to have knowledge of basic 20th century mathematics and physics.

You will start by building a very simple machine, and then you will extend this machine through several stages, so that by the end you will have a fully functional machine.

When you have finished the machine, you are not quite ready to run programs yet.
First you will need to install in the machine an initial program, which is supplied here.

\comment{Provide time estimates for the above tasks.}

\section{Building the Machine}
\label{sec:building-machine}

By following the instructions here, you can develop the machine through several stages, each stage leading to a more capable machine than the preceding one.

Number systems used here include decimal, binary, and hexadecimal.
All non-decimal numbers are explictly indicated by subscripts indicating the number base in decimal.
Further detail on binary and hexadecimal notation can be found in Appendixes~\ref{sec:binary-notation} and \ref{sec:hexadecimal-notation}.

\subsection{Storage}

To build the machine, you first need storage to represent state consisting of binary digits (bits).
A storage \emph{element} is a group of bits forming a unit.
Each element has an identity, so that it can be uniquely referred to, and the contents can retrieved or stored in one atomic operation.
This machine employs element sizes of 1, 8, and 64 bits.
The storage is divided into three categories:
\begin{description}
\item[Memory] 
The memory consists of a number of 8-bit elements, up to a maximum of $2^{64}$.
Each memory element can be referred to using an integer in the range from $0$ to $2^{64} - 1$.
\item[Registers] 
There are two registers: the \PC and the \SP.
Both of these are 64-bit elements.
The contents of either register are interpreted as a non-negative integer referring to a memory element.
\item[Flags] 
There is one flag: the \TERM flag, which is a 1-bit element.
This bit represents a value of \T or \F.
The machine will stop if and only if this is set to \T.
\end{description}
We need to define a few terms to talk about storage operations.
\begin{itemize}
\item ``The value of an element'' means the contents as retrieved from that element.
\item ``To set the value of an element to $x$'' means to store the value of $x$ in that element.
\item ``To increment an element by $n$'' means to set the value of $x$ to the value of the element and then set that element to $x + n$.
\item ``To decrement an element by $n$'' means to set the value of $x$ to the value of the element and then set that element to $x - n$.
\item ``The value of byte $n$ in $x$'' means the contents of the 8-bit subsequence of $x$ from bit number $8n+7$ to bit number $8n$.
\item ``To set byte $n$ in $x$ to $y$'' means to store the 8-bit sequence $y$ in the 8-bit subsequence of $x$ from bit number $8n+7$ to bit number $8n$, while keeping all other bits of $x$ unchanged.
\end{itemize}
Typically these would be basic operations in your implementation.

\comment{Describe tests to verify that the storage is working correctly.}

\subsection{A Basic Machine}

When the storage has been implemented and is working correctly, you can proceed with implementing a basic version of the machine.
For practical reasons we will build this version of the machine with a fixed memory size of $2^{8}$ elements.

Machine operation proceeds through two phases: first the storage is initialized, and then the main procedure is executed.

\subsubsection{Storage Initialization}

To initialize the storage, you need to set registers and flags to predefined values.
The \PC is set to the address of the first memory element, that is, $0$.
The \SP is set to the first element beyond the range of addressable elements, that is $2^{8}$.
The values are given in the table below.

\begin{center}
  \begin{tabular}{@{}ll@{}}
    \hline
    Register or flag & Value                   \\
    \hline
    \PC              & \hex{0000000000000000}  \\
    \SP              & \hex{0000000000000100}  \\
    \TERM            & \F                      \\
    \hline
  \end{tabular}
\end{center}

\subsubsection{The Main Procedure}

When you have initialized the storage, simply execute the main procedure repeatedly until the value of the \TERM flag is \T.
The main procedure must carry out the following steps.
\begin{enumerate}
\item Set the 64-bit value $p$ to the value of \PC.
\item Increment \PC.
\item Retrieve the 8-bit value of memory element $p$.
\item If that value is
  \begin{itemize}
  \item \EXIT{}, then set the value of  \TERM to \T.
  \item \NOP{}, then do nothing.
  \item \JUMP{}, then do as follows:
    \begin{enumerate}
    \item Set the value of $a$ to 0.
    \item Set the value of $i$ to 0.
    \item Do the following 8 times:
      \begin{enumerate}
      \item Set byte $i$ of $a$ to memory element \SP.
      \item Increment \SP by 1.
      \item Increment $i$ by 1.
      \end{enumerate}
    \item Set \PC to $a$.
    \end{enumerate}
  \item \PUSHL{}, then do as follows:
    \begin{enumerate}
    \item Decrement \PC by 8.
    \item Set the value of $s$ to the value of \SP.
    \item Do the following 8 times:
      \begin{enumerate}
      \item Set the value of memory element $s$ to the value of memory element \PC.
      \item Increment \PC by 1.
      \item Increment $s$ by 1.
      \end{enumerate}
    \end{enumerate}
  \end{itemize}
\end{enumerate}

\subsubsection{Testing the Basic Machine}

To test the machine, first initialize all memory elements to the value \hex{00}, and then set the following memory elements to the provided values.

\begin{center}
  \begin{tabular}{@{}ll@{}}
    \hline
    Memory element         & Value \\
    \hline
    \hex{0000000000000000} & \hex{01} \\
    \hex{0000000000000001} & \hex{0B} \\
    \hex{0000000000000002} & \hex{0F} \\
    \hex{0000000000000003} & \hex{00} \\
    \hex{0000000000000004} & \hex{00} \\
    \hex{0000000000000005} & \hex{00} \\
    \hex{0000000000000006} & \hex{00} \\
    \hex{0000000000000007} & \hex{00} \\
    \hex{0000000000000008} & \hex{00} \\
    \hex{0000000000000009} & \hex{00} \\
    \hex{000000000000000A} & \hex{02} \\
    \hline
  \end{tabular}
\end{center}

Now start the machine.
When it terminates, all memory elements should remain unchanged, except the value of memory element \hex{00000000000000F8}, which should change to \hex{0F}.
The registers and flags should have the following values:

\begin{center}
  \begin{tabular}{@{}ll@{}}
    \hline
    Register or flag & Value                   \\
    \hline
    \PC              & \hex{0000000000000010}  \\
    \SP              & \hex{0000000000000100}  \\
    \TERM            & \T                      \\
    \hline
  \end{tabular}
\end{center}


If the machine does not terminate or you did not get exactly these results, you must have made a mistake somewhere.
Do not proceed with the next step until you have rectified the mistake and obtained the correct results.

\subsection{A More Capable Machine}

TODO

\section{Building the Devices}
\label{sec:building-devices}

At this point you should have a machine with fully functional computational capabilities.
What's missing are the devices that allow the machine to consume data from its environment and to produce data to its environment.
There are four devices:
\begin{itemize}
\item The \emph{Image Input} device allows input of images as a two-dimensional matrix of gray-scale values.
  This device is very important, because it is what enables the machine to load programs encoded as images on the film.
\item The \emph{Image Output} device allows output of color images.
  Moving images are supported by outputting a time series of still images.
\item The \emph{Audio Output} device allows output of audio signals as a time series of amplitude values.
\item The \emph{Text Output} device allows output of text as a stream of characters.
\end{itemize}
The descriptions below will only explain the correspondence between machine events and real world events.
It is up to you to make sure that the interpretation of real world events is faithfully implemented.

\subsection{Image Input}

\subsection{Image Output}

\subsection{Audio Output}

\subsection{Text Output}

\section{Installing the Initial Program}

\appendix

\section{Binary Notation}
\label{sec:binary-notation}

Each storage element contains a binary value, that is, a sequence of binary digits that can be retrieved or modified in one atomic operation.

In this documentation, individual binary digits are referred to using non-negative integers, listed in decreasing order.  For example,
\begin{itemize}
\item In the binary number \bin{1000}, bit number 3 is 1.
\item In the binary number \bin{0001}, bit number 0 is 1.
\end{itemize}

When binary values represent integers, each bit represents two to the power of the bit number.
In other words, bit number 0 is always the least significant bit.

\section{Hexadecimal Notation}
\label{sec:hexadecimal-notation}

For convenience, binary numbers are normally written in hexadecimal notation.
Each hexadecimal digit corresponds to a group of four binary digits, as shown in the following table.

\begin{center}
  \begin{tabular}{@{}ll@{}}
    \hline
    Hexadecimal digit & Group of binary digits \\
    \hline
    \num{0}           & \num{0000}   \\
    \num{1}           & \num{0001}   \\
    \num{2}           & \num{0010}   \\
    \num{3}           & \num{0011}   \\
    \num{4}           & \num{0100}   \\
    \num{5}           & \num{0101}   \\
    \num{6}           & \num{0110}   \\
    \num{7}           & \num{0111}   \\
    \num{8}           & \num{1000}   \\
    \num{9}           & \num{1001}   \\
    \num{A}           & \num{1010}   \\
    \num{B}           & \num{1011}   \\
    \num{C}           & \num{1100}   \\
    \num{D}           & \num{1101}   \\
    \num{E}           & \num{1110}   \\
    \num{F}           & \num{1111}   \\
    \hline
  \end{tabular}
\end{center}

\section{The Initial Program}
\label{sec:initial-program}

\comment{Include a listing in hex of the initial program.}
\end{document}
